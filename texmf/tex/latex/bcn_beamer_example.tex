% ~~~~~~~~~~~~~~~~~~~~~~~~~~~~~~~~~~~~~~~~~~~~~~~~~~~~~~~~~~~~~~~~~~~~~~~~~~~~~~ 
%
% bcn_beamer_example.tex - An customizable example file for bcn_letter.lco
%
% Copyright (C)     2014-01-16    Bijan Chokoufe Nejad    <bijan@chokoufe.com>
% Recent versions:  https://github.com/bijanc/bcn_scripts
%
% This source code is free software; you can redistribute it and/or
% modify it under the terms of the GNU General Public License Version 2: 
% http://www.gnu.org/licenses/gpl-2.0-standalone.html
%
% ~~~~~~~~~~~~~~~~~~~~~~~~~~~~~~~~~~~~~~~~~~~~~~~~~~~~~~~~~~~~~~~~~~~~~~~~~~~~~~ 
\documentclass[mathserif]{beamer}
\usepackage{bcn_beamer}
\usepackage[english]{babel}
\usepackage[babel]{csquotes} %Sets the ``'', '', `` correctly with biber/biblatex
\usepackage[style=alphabetic, %numeric-comp is simple numbers
  ]{biblatex}
\mode<presentation>

%==============================================================================%
%                                STYLE SETTINGS                                %
%==============================================================================%
\setbeamercovered{transparent}
\usetheme{default}
\newcommand\animation{}
\newcommand\bad[1]{{\color{red}#1}}
\newcommand\good[1]{{\color{\sndcol}#1}}
\usepackage[T1]{fontenc}
\usepackage{lmodern}
\newcommand\fstcol\myorange
\newcommand\sndcol\mygreenblue

% Enable this to set background_file as transparent background
%\usebackgroundtemplate{\transparent{0.06} 
%\includegraphics[height=\paperheight]{background_file}
%}

\setbeamertemplate{blocks}[rounded][shadow=true]
\setbeamertemplate{navigation symbols}{}
\setbeamertemplate{frametitle}[default][right]
\setbeamertemplate{enumerate item}{(\alph{enumi})}
\setbeamertemplate{footline}{
  \hs{.5}\vs{1} % Ensuring enough space for the foot line
}

%==============================================================================%
%                             GENERAL INFORMATION                              %
%==============================================================================%
\title{Title of this talk}
\author{Max Mustermann}
\institute{University of Mustercity}
\date{September 20, 2013}

%==============================================================================%
%                              SLIDES OF THE TALK                              %
%==============================================================================%
\begin{document}

\begin{frame}
  \titlepage
\end{frame}

% A simple footer
\setbeamertemplate{footline}{{}{ \hfill\insertframenumber
  /\inserttotalframenumber\hskip 1em }
}

\begin{Bframe}{Outline}
  \tableofcontents
\end{Bframe}

% A simple footer, now with navigation bar
\setbeamertemplate{footline}{
  \insertsectionnavigationhorizontal{\textwidth}{}{ \hfill\insertframenumber
  /\inserttotalframenumber\hskip 1em }
}

%==============================================================================%%
\beamersection{First Important Section}

\begin{Bframe}
  {Motivation}
  \begin{itemize}
    \item Give some motivation \vs 1
    \item It generally helps \vs 1
    \item But sometimes not
  \end{itemize}
\end{Bframe}

\begin{Bframe}
  {Big pictures}
  % Includes some pictures
  %\includegraphics[width=\textwidth]{pics/schematic.pdf}
\end{Bframe}

\begin{Bframe}
  {Columns}
  \begin{columns}
    \column{.5\textwidth}
      %\includegraphics[width=\textwidth]{pics/some_pic.pdf}
    \column{.5\textwidth}
      Columns allow to put more informations on a slide and structure the slide.
      \vs 1

      In a proper unit system, you have
      \begin{Balign}
        E = m 
      \end{Balign}
  \end{columns}
\end{Bframe}

%==============================================================================%%
\beamersection{Second Important Section}

\begin{Bframe}
  [Subtitle]{Title}
  The second part
  \vs 1\pause

  is a bit \bad{short}! But then coffee break is \good{earlier}!
\end{Bframe}

%==============================================================================%%
\begin{Bframe}
  {Conclusions/Outlook}
  \begin{itemize}
    \item This should give you the basic tools for a decent tools. \vs 1
    \item Remember: Less is more.
  \end{itemize}
\end{Bframe}

% Enable Reference slide here
%\begin{Bframe}
  %{References}
%\renewcommand*{\bibfont}{\footnotesize}
  %\printbibliography
%\end{Bframe}
\end{document}
